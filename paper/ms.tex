\documentclass[preprint]{aastex61}
\usepackage{microtype}
\usepackage{url}
\usepackage{amsmath}
\usepackage{amssymb}
\usepackage{natbib}
\usepackage{multirow}

\pdfoutput=1
\bibliographystyle{aasjournal}

\include{vc}

%%%%%%%%%%%%%%%%%%%%%%%%%%%%%%%%%%%%%%%%%%%%%%%%%%
% Useful aliases
\newcommand{\project}[1]{\textsf{#1}}
\newcommand{\TODO}[1]{{\it \color{red} (#1)}}

%%%%%%%%%%%%%%%%%%%%%%%%%%%%%%%%%%%%%%%%%%%%%%%%%%
%\slugcomment{Submitted to the Astronomical Journal (AJ)}
\shorttitle{Legacy Survey Large Galaxy Atlas}
\shortauthors{Moustakas, Lang, et al.}

\begin{document}

\title{Legacy Survey Large Galaxy Atlas}

\author{John Moustakas}
\affil{Department of Physics \& Astronomy, Siena College, 515 Loudon Road,
  Loudonville, 12211}
\email{jmoustakas@siena.edu}

\author{Dustin Lang} \affil{Dunlap Institute for Astronomy \& Astrophysics and
  Department of Astronomy \& Astrophysics University of Toronto}
\email{dstndstn@gmail.com}

\begin{abstract}
I like cheese.
\end{abstract}

\input{intro}

\input{sample}

\input{modeling}

\input{discussion}

\begin{itemize}

\item{Look at van Dokkum, Conroy et al. (2017) -- the central regions of massive
  galaxies really seem to be special, with the IMF slope varying as a function
  of velocity dispersion $\sigma$.  Connect this with the two-phase formation
  paradigm for massive galaxies, where the central regions form first and then
  grow inside-out.  What are the implications here on our SED modeling for the
  integrated stellar mass function?  For example, consider a bursty star
  formation history, bottom heavy IMF for the central region and then a more
  quiescent, Salpeter-like IMF for the outer regions.}

\item{Think about how AGB stars will affect the WISE photometry.}

\end{itemize}


\input{summary}

%\bibliographystyle{apj}
%\bibliography{/Users/ioannis/bibdesk/ioannis}
%\input{target-truth.bbl}

\end{document}


